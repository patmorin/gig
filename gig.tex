\documentclass{patmorin}
\listfiles
\usepackage{pat}
\usepackage{paralist}
\usepackage[T1]{fontenc}
\usepackage[utf8]{inputenc}
\usepackage{bbm}  % needed for \mathbbm{1}
% \usepackage{logix}
\usepackage{halloweenmath}
\usepackage{stmaryrd}

\usepackage{todonotes}
\usepackage{tcolorbox}
\usepackage{booktabs}
\usepackage{multirow}
\usepackage{comment}

\usepackage{thm-restate}


% etoolbox allows for robust commands that don't need \protect, e.g.
% \newrobustcmd{\onesub}{\mathord{\includegraphics{figs/one-sub}}}
% \subsection{Approximate Voronoi Diagrams in $G^{\onesub}_k$}
\usepackage{etoolbox}

% david proposes the following additions
\renewcommand{\ge}{\geqslant}
\renewcommand{\le}{\leqslant}
\renewcommand{\geq}{\geqslant}
\renewcommand{\leq}{\leqslant}

\newcommand{\david}[1]{{\color{orange} David: #1}}
\newcommand{\vida}[1]{{\color{DarkGreen} Vida: #1}}
\newcommand{\pat}[1]{\textcolor{Blue}{Pat: #1}}
\newcommand{\gwen}[1]{\textcolor{Purple}{Gwen: #1}}
\newcommand{\piotr}[1]{\textcolor{red}{Piotr: #1}}

% \numberwithin{equation}{lem}


\newenvironment{clmproof}{\noindent\emph{Proof of Claim:}}{\hfill\rule{1ex}{1ex}}

\usepackage[longnamesfirst,numbers,sort&compress]{natbib}

\usepackage[mathlines]{lineno}
\setlength{\linenumbersep}{2em}
% \linenumbers
% \rightlinenumbers
% \linenumbers
\newcommand*\patchAmsMathEnvironmentForLineno[1]{%
 \expandafter\let\csname old#1\expandafter\endcsname\csname #1\endcsname
 \expandafter\let\csname oldend#1\expandafter\endcsname\csname end#1\endcsname
 \renewenvironment{#1}%
    {\linenomath\csname old#1\endcsname}%
    {\csname oldend#1\endcsname\endlinenomath}}%
\newcommand*\patchBothAmsMathEnvironmentsForLineno[1]{%
 \patchAmsMathEnvironmentForLineno{#1}%
 \patchAmsMathEnvironmentForLineno{#1*}}%
\AtBeginDocument{%
\patchBothAmsMathEnvironmentsForLineno{equation}%
\patchBothAmsMathEnvironmentsForLineno{align}%
\patchBothAmsMathEnvironmentsForLineno{flalign}%
\patchBothAmsMathEnvironmentsForLineno{alignat}%
\patchBothAmsMathEnvironmentsForLineno{gather}%
\patchBothAmsMathEnvironmentsForLineno{multline}%
}



% Taken from
% https://tex.stackexchange.com/questions/42726/align-but-show-one-equation-number-at-the-end
\newcommand\numberthis{\addtocounter{equation}{1}\tag{\theequation}}

\definecolor{brightmaroon}{rgb}{0.76, 0.13, 0.28}
\definecolor{linkblue}{rgb}{0, 0.337, 0.227}
\newcommand{\defin}[1]{\emph{\textcolor{brightmaroon}{#1}}}
\makeatletter
\def\mathcolor#1#{\@mathcolor{#1}}
\def\@mathcolor#1#2#3{%
  \protect\leavevmode
  \begingroup
    \color#1{#2}#3%
  \endgroup
}
\makeatother
\newcommand{\mathdefin}[1]{\mathcolor{brightmaroon}{#1}}
% \newcommand{\mathdefin}[1]{\color{brightmaroon}#1}}
\setlength{\parskip}{1ex}

% Document-specific commands and math operators
\DeclareMathOperator{\tw}{tw}
\DeclareMathOperator{\pw}{pw}
\DeclareMathOperator{\bw}{bw}
\DeclareMathOperator{\td}{td}
%\DeclareMathOperator{\rtw}{rtw}
\DeclareMathOperator{\diam}{diam}
\DeclareMathOperator{\mindist}{min-dist}
\DeclareMathOperator{\dist}{dist}
\DeclareMathOperator{\ld}{ld}
\DeclareMathOperator{\polylog}{polylog}
\DeclareMathOperator{\evol}{Evol}
\DeclareMathOperator{\ivol}{Ivol}
\DeclareMathOperator{\tvol}{Tvol}
\newcommand{\NN}{\mathbb{N}}
\newcommand{\GG}{\mathcal{G}}

\title{\MakeUppercase{\boldmath Geometric Intersection Graphs and Blowups of Bounded Pathwidth Graphs}}

%\title{\MakeUppercase{\boldmath Planar graphs are contained in $\tilde{O}(\sqrt{n})$-blowups of fans}}

%Fan-Partitions of Planar Graphs (and Beyond)  \newline by Local Sparsification and Volume-Preserving Embeddings}}

\author{TBD}
 % Vida Dujmovi{\'c}\,\footnote{School of Computer Science and Electrical Engineering, University of Ottawa, Ottawa, Canada (\texttt{vida.dujmovic@uottawa.ca}). Research supported by NSERC and a University of Ottawa Research Chair.}
 % \qquad
 % Gwena\"el Joret\footnote{D\'epartement d'Informatique, Universit\'e libre de Bruxelles, Belgium ({\tt gwenael.joret@ulb.be}). G.\ Joret is supported by the Belgian National Fund for Scientific Research (FNRS) and by the Australian Research Council.}
 % \qquad
 % Piotr Micek\footnote{Department of Theoretical Computer Science, Jagiellonian University, Kraków, Poland (\texttt{piotr.micek@uj.edu.pl}). Research supported
 % the National Science Center of Poland under grant UMO-2018/31/G/ST1/03718 within the BEETHOVEN program.}
 % \qquad
 % Pat Morin\footnote{School of Computer Science, Carleton University, Ottawa, Canada (\texttt{morin@scs.carleton.ca}). Research supported by NSERC and the Ontario Ministry of Research and Innovation.}
 % \qquad
 % David~R.~Wood\footnote{School of Mathematics, Monash University, Melbourne, Australia (\texttt{david.wood@monash.edu}). Research supported by the Australian Research Council.}
 % }

\date{}


\begin{document}

\maketitle

\section{Introduction}


\section{A First Lemma}

For a point $p\in\R^2$ and a real number $r$, let $\mathdefin{C(p,r)}:=\{q\in\R^2: d_2(p,q)=r\}$ be the circle of radius $r$ centered at $p$ and $\mathdefin{D(p,r)}:=\{q\in\R^2: d_2(p,q)\le r\}$ be the closed disc of radius $r$ centered at $p$.  Let $\mathbf{0}:=(0,0)$ denote the origin.

\begin{lem}
  Let $S$ be a set of $n$ pairwise-disjoint discs in the plane, and let $v\in C(\mathbf{0},1)$ be a uniformly random unit vector.  Then the expected number of disks in $S$ that intersect $C(v/2,1)$ is $O(\sqrt{n\log n})$.
\end{lem}

In the following proof, we repeatedly use the fact that, for concave increasing $f:\R_\ge 0\to\R$ and non-negative numbers $n_1,\ldots,n_k$ that sum to $n$, the sum $\sum_{i=1}^k f(n_i)$ is maximized when $n_1=\cdots=n_k=n/k$.  That is, $\sum_{i=1}^k f(n_i) \le k\,f(n/k)$.  In particular, we use this for $f(x)=\sqrt{x}$, in which case $\sum_{i=1}^k \sqrt{n_i}\le k\sqrt{n/k}=\sqrt{kn}$.

\begin{proof}
  The circle $C_i:=C(v/2,1)$ is contained in the disc $D:=(\mathbf{0},3/2)$, so we may assume that every disc in $S$ intersects $D$.  Since the discs in $S$ are pairwise disjoint, a standard packing argument implies that $S$ contains $O(1)\subseteq O(\sqrt{n})$ discs of radius greater than $1$, so we assume that all discs in $S$ have radius at most $1$.

  For any integer $i\ge 1$, let $r_i=2^{-i}$ and let $S_i$ be the set of discs in $S$ whose radius is in the interval $(r_i,2r_i]$. For each $i\ge 1$, let $n_i:=|S_i|$. Fix some disc $B\in S_i$ with center $c$.  If $B$ intersects $C(v/2,1)$, then $v/2$ is contained in the annulus $A(c,1+2r_i,1-2r_i)$.  Some trigonometry and the inequality $1-\cos(\alpha)\le \alpha^2$ shows that the intersection of $C(\mathbf{0},1/2})$ with $A(c,1+2r_i,1-2r_i)$ is the union of (at most 2) circular arcs, each of length $\Omega(r_i)$ and having total length  $O(\sqrt{r_i})$. Therefore, the probability that $B$ intersects $C(v/2,1)$ is $O(\sqrt{r_i})$.  Therefore, the expected number of discs in $S$ of radius at most $r_i$ that intersect $C(v/2,1)$ is $O(n\sqrt{r_i})$.  This quantity is $O(\sqrt{n})$ for $r_i\le 2/n$, so we assume that all discs in $S$ have radius at least $1/n$.  In particular, $S_i$ is empty for all $i \ge \log_2 n$.\todo{Is this necessary?}

  We now push this argument further to consider a set of disks in $S_i$ that each intersect a ray through the origin. Fix a ray $R$ originating at $\mathbf{0}$ and consider the set $S_{i,R}$ of discs in $S_i$ that intersect $R$. Let $n_{i,R}:=|S_{i,r}|$.  Then we can partition $S_{i,R}$ into $\alpha\in O(1)$ sets $S_{i,r,1},\ldots,S_{i,r,\alpha}$ such that, for any $p\in C(\mathbf{0},1/2)$,  $C(p,1)$ intersects at most one disc in $S_{i,R,a}$ for each $a\in[\alpha]$.  Let $n_{i,R,a}:=|S_{i,R,a}|$. For each $a\in[\alpha]$ the union of the $(1+2r_i,1-2r_i)$-annuli centered at the centers of discs in $S_{i,R,a}$ contains a portion of $C(\mathbf{0},1/2)$ whose length is $O(\sqrt{n_{i,R,a}r_i})$.  Therefore, $C(v/2,1)$ intersects at most one disc in $S_{i,R,a}$ and this intersection occurs with probability $O(\sqrt{n_{i,R,a}r_i})$.  Therefore, the expected number of discs in $S_{i,R,a}$ intersected by $C(v/2,1)$ is $O(\sqrt{n_{i,R,a}r_i})$.  Therefore, the expected number of discs in $S_{i,R}$ intersected by $C(v/2,1)$ is
  \[
    \sum_{a=1}^\alpha O(\sqrt{n_{i,R,a}r_i}) \subseteq O(\alpha\sqrt{n_{i,R}r_i/\alpha}) = O(\sqrt{\alpha n_{i,R}r_i}) = O(\sqrt{n_{i,R}r_i}) \enspace .
  \]
  Define a set of $p\in O(1/r_i)$ rays $R_1,\ldots,R_p$ such that $\bigcup_{\ell=1}^p S_{i,R_\ell}$ covers $S_i$. Remove each disc in $S_i$ from all but one $S_{i,R_j}$ so that $S_{i,R_1},\ldots,S_{i,R_p}$ is a partition of $S_i$.  Then the expected number of discs in $S_i$ intersected by $C(v/2,1)$ is
  \[
    \sum_{\ell=1}^p O(\sqrt{n_{i,R_p}r_i}) = O(p\sqrt{nr_i/p}) = O(\sqrt{n_ipr_i}) = O(\sqrt{n_i}) \enspace .
  \]
  Finally, the expected number of discs intersected by $C_j$ is
  \[
    \sum_{i=1}^{\log_2 n} O(\sqrt{n_i}) = O(\log n\, \sqrt{n/\log n}) = O(\sqrt{n\log n}) \enspace . \qedhere
  \]
\end{proof}










\end{document}
